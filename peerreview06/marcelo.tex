% Created 2017-11-09 qui 14:08
% Intended LaTeX compiler: pdflatex
\documentclass[11pt]{article}
\usepackage[utf8]{inputenc}
\usepackage[T1]{fontenc}
\usepackage{graphicx}
\usepackage{grffile}
\usepackage{longtable}
\usepackage{wrapfig}
\usepackage{rotating}
\usepackage[normalem]{ulem}
\usepackage{amsmath}
\usepackage{textcomp}
\usepackage{amssymb}
\usepackage{capt-of}
\usepackage{hyperref}
\author{gil}
\date{\today}
\title{Peer Review Escrita 6}
\hypersetup{
 pdfauthor={gil},
 pdftitle={},
 pdfkeywords={},
 pdfsubject={},
 pdfcreator={Emacs 25.1.1 (Org mode 9.0.5)},
 pdflang={English}}
\begin{document}

Autor: Marcelo Andrade

Avaliador: Pedro Cunial

\section{Existe um entendimento convincente do contexto social em que se quer analisar?}
Apesar de demonstrar grande conhecimento sobre análise de dados e sua
manipulação, o autor não demonstra grande profundidade na análise do contexto
social no qual o texto se constrói (jazz nos anos 1960 até 1970, passando por
subgeneros como bebop, fusion etc), limitando esta explicação de contexto a
introdução e o primeiro parágrafo da seção 2 (Método Utilizado),
no entanto, com o decorrer do texto, isso não parece ser um problema.

\section{Os conceitos e métricas de Análise de Redes Sociais foram devidamente justificados?}
De maneira geral, o autor demonstra domínio das técnicas de análise de dados e
redes sociais, no entanto, o conceito de /emph{smallworldness} fica um pouco em
aberto e confuso, não deixando claro para o leitor o que ele realmente avalia e
porquê seria uma boa medida para a análise realizada.


\section{Os métodos foram explicados de forma clara?}
Assim como nas respostas anteriores, o autor demonstra grande domínio técnico,
tal que as decisões tomadas quanto a técnica de análise parecem justificáveis,
no entanto, não há justificativa para um público mais leigo (ou que não tenha um
conhecimento mais avançado de estatística e redes sociais). Ao mesmo tempo,
acredito que pela presença de referências técnicas sobre os métodos utilizados
seja o suficiente de maneira geral.

\section{O texto é acompanhado de anexos com análises clara e que suportam de
  forma inequívoca as afirmações do texto?}
Sim! O texto inteiro é justificado de maneira científico-matemática, claramente
esse não é um problema para o autor.

\section{A interpretação dos resultados foi realizada de forma adequada?}
De maneira geral, a análise e interpretação dos resultados parece bastante
correta. O fato do autor ter se baseado tanto em técnicas matemáticas e de
estatística as torna quase que irrefutáveis, o que me pareceu bom e convincente.

\section{A narrativa do texto oferece ``começo, meio e fim''? É possível
  observar o encadeamento lógico das ideias?}
Acredito que em questão de estilo de texto e desenvolvimento de ideias o autor
tenha se destacado bastante. Apesar do texto ser extremamente técnico, a leitura
não é massante as ideias estão bastante bem organizadas (o que é bastante raro
nas leituras do tipo).

\end{document}